%% Preamble
\documentclass{beamer}
%\usepackage{fontspec}
%\usetheme[beteckna]{Median}
\usepackage[T1]{fontenc}
\usepackage[black]{merriweather}
\usepackage{sansmath}
\usepackage{tikz}
\usetheme{Median}

%% Macros
\renewcommand{\l}{\lambda} % We don't really care about this diacritic
\newcommand{\aeq}{\eq_\alpha}
\newcommand{\ato}{\to_\alpha}
\newcommand{\beq}{\eq_\beta}
\newcommand{\bto}{\to_\beta}

%% Title and other silliness
\title{An Introduction to the $\l$-calculus}
\author{Toli Paine}
\institute{Quantcast}
\date{April 2014}

\AtBeginSection {
    \begin{frame}[plain]
        \frametitle{\\[2em]}
        \tableofcontents[currentsection]
    \end{frame}
}

% We're going to want this later
%\[
%    y \equiv \l f.(\l x.f (x x)) (\l x.f (x x))
%\]

%% Slides
\begin{document}
    \frame{\titlepage}

    \section{Introduction}
    \begin{frame}
        \frametitle{Why the $\l$-calculus?}
        \pause
        \begin{itemize}[<+->]
            \setlength{\itemsep}{1.5em}
            \item Simple, yet powerful: encompasses all of computing
                (Turing-complete!)
            \item Much easier to understand than Turing machines
            \item Provides the basis of modern ``functional programming''
                languages \\[0.25em]
                \begin{itemize}[<+->]
                    \item Scala, Clojure, Scheme, Haskell
                \end{itemize}
                \vspace{-0.5em}
            \item Ever wondered what the hell ``Y Combinator'' actually means?
        \end{itemize}
    \end{frame}
    %TODO: add pictures! Church, Turing, etc.
    \begin{frame}
        \frametitle{The Church-Turing Thesis}
    \end{frame}
    \begin{frame}
        \frametitle{Definitions}
    \end{frame}

    \section{Boolean Logic}
    \begin{frame}
        \frametitle{Encoding truth values}
    \end{frame}
    \begin{frame}
        \frametitle{Boolean operators}
    \end{frame}

    \section{Numbers!}
    \begin{frame}
        \frametitle{Church's encoding (for natural numbers)}
    \end{frame}

    \section{Data Structures}
    \begin{frame}
        \frametitle{Encoding linked lists}
    \end{frame}

    \section{Combinators and Recursion}
    \begin{frame}
        \frametitle{How do we build recursive functions?}
    \end{frame}

    \section{Conclusion}
    \begin{frame}
        \frametitle{Further reading}
        \begin{itemize}
            \setlength{\itemsep}{1.5em}
            \item Combinatory logic
            \item Curry-Howard isomorphism
            \item System F
            \item Hindley-Milner type system
        \end{itemize}
    \end{frame}
\end{document}
